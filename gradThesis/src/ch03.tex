\chapter{拓扑物理基本概念}
拓扑物理有一个重要的概念——Berry phase,可以从含时薛定谔方程的求解过程中生成。因此本部分首先会简要概括含时薛定谔方程的求解过程,然后引入Berry phase,最后定义与之相关的其他拓扑守恒量。
\section{含时薛定谔方程的求解}
通常情况下系统的波函数会随时间演化。因此我们可以给波函数添加时间标签
\begin{equation}
  \ket{\alpha, t_0; t}, \qquad (t > t_0)
\end{equation}
其中$t_0$是初始时间。系统在初始时间的波函数也可以被简写为$\ket{\alpha, t_0}$,。
  时间演化算子被定义用来描述某时刻的波函数和另一时刻的关系,即
  \begin{equation}
    \ket{\alpha, t_0; t} = \mathscr{U}(t, t_0)\ket{\alpha, t_0}
    \label{defineEvoOpt}
  \end{equation}
时间演化算子应该满足一些约束条件。首先我们希望系统在演化过程中概率是守恒的,即
  \begin{equation}
    \braket{\alpha, t_0 | \alpha, t_0} = 1 \Rightarrow \braket{\alpha, t_0; t | \alpha, t_0; t} = 1
  \end{equation}
把上式插入(\ref{defineEvoOpt}),可以知道$\mathscr{U}$是酉算子
\begin{equation}
  \mathscr{U}^\dagger(t, t_0)  \mathscr{U}(t, t_0) = 1
\label{EvoTUni}
\end{equation}
第二点,$\mathscr{U}$应该可以被切割为不同的时间段
\begin{equation}
\mathscr{U}(t_2, t_0) = \mathscr{U}(t_2, t_1) \mathscr{U}(t_1, t_0)
\label{EvoTSlice}
\end{equation}
有这两个约束,我们就可以断定当时间间隔$dt$很小时,$\mathscr{U}$可以被写为
\begin{equation}
\mathscr{U}(t + dt, t) = 1 - i \Omega dt
\end{equation}
其中$\Omega$是厄米算子。显然这个形式是符合约束(\ref{EvoTUni}) (\ref{EvoTSlice})的。
$\mathscr{U}$算子的作用是把一个波函数映射到另一个波函数,因此它本身是没有量纲的。
这就表明$\Omega$具有时间倒数,或者说频率,的量纲。另一方面,含时薛定谔方程的特殊
之处就是它是哈密顿量含时间因子。换句话说,哈密顿量是推动波函数的量,和时间演化算
子息息相关。而能量又可以写作$E = \hbar \omega$。因此可以(正确的)猜想
\begin{equation}
  \Omega = \frac{H}{\hbar}
\end{equation}
总之,$\mathscr{U}$可以写作
\begin{equation}
  \mathscr{U}(t + dt, t) = (1 - \frac{iH dt}{\hbar}) \mathscr{U}(t, t_0)
\end{equation}
可以得到时间演化算子的薛定谔方程
\begin{equation}
  i\hbar\frac{\partial}{\partial t} \mathscr{U}(t, t_0) = H\mathscr{U}(t, t_0)
\end{equation}
对于不同类型的$H(t)$,这个方程有不同的解。假如不同时间的$H(t)$可以对易,那么解可以写为
\begin{equation}
  \mathscr{U}(t, t_0) = exp[-\frac{i}{\hbar} \int_{t_0}^{t} dt' H(t') ]
\end{equation}
如果能得到$H$的一组基
\begin{equation}
  H \ket{a, t} = E_a \ket{a, t}
\end{equation}
那么单个基的含时间波函数就可以写作
\begin{equation}
  \ket{\phi(t)} = exp[-\frac{i}{\hbar} \int_{t_0}^{t} dt' E_a(t') ] \ket{a, t}
\end{equation}

\section{拓扑不变量}

实际上$H$会比上面所述的情况复杂。例如给体系加入一个会变化方向的磁场,不同时刻的$H$则不对易。但我们仍然可以从上面的解得到灵感来猜测一个通解
\begin{equation}
  \ket{\phi(t)} = exp[- i \gamma_a(t) -\frac{i}{\hbar} \int_{t_0}^{t} dt' E_a(t') ] \ket{a, t}
\end{equation}
而$\gamma_{a}(t)$是引入的新相位。通常情况下$H$是由运动轨迹简介和时间想联系,因此这里把$E_a(t)$改成$E(\mathbf{R}(t))$。把上式代回到含时薛定谔方程可以得到$\gamma_a(t)$的约束
\begin{equation}
  \gamma_a(t) = i \int_{\mathbf{R}(0)}^{\mathbf{R}(t)}{d\mathbf{R}} A_a(\mathbf{R})
\end{equation}
$\gamma_a(t)$被称作Berry phase。被积分函数为
\begin{equation}
  A_a(\mathbf{R}) = \braket{a, \mathbf{R}(t) | \nabla_{\mathbf{R}} | a, \mathbf{R}(t)}
\end{equation}
则为Berry connection. 当例子沿闭合轨道运动(这个常常和周期性边界有关),上式为闭合路径积分,可以用斯托克斯公式重写为
\begin{equation}
  \gamma_a = \int_{S}{d\mathbf{S}} \cdot \Omega_a(\mathbf{R})
\end{equation}
其中Berry curvature为
\begin{equation}
\Omega_a(\mathbf{R}) = \nabla_{\mathbf{R}} \times A_a(\mathbf{R})
\end{equation}
其他重要的量也可以用Berry connection表示。例如霍尔电导
\begin{equation}
  \sigma_{xy} = \frac{e^2}{h}C_1
  \label{hallInduc}
\end{equation}
$C_1$为Chern number
\begin{equation}
  C_1 = \frac{1}{2 \pi} \int{d^2k}(\nabla \times \mathbf{A})
\end{equation}
如果处理的是晶格系统,Berry connection的波函数就被替换为布洛赫波函数即可。

Chern number是一个基本的拓扑不变量。具有不同的Chern number的材料会有不同的拓扑性质。下个部分包含几个低拓扑模型的例子。它们都和Chern number的丰富物理内涵有关:1. 离散的霍尔电导; 2. 拓扑相变; 3. 拓扑绝缘体的Edge mode。